\documentclass[12pt]{article}

\usepackage[magyar]{babel}
%\usepackage{t1enc}
%\usepackage[latin2]{inputenc}
\usepackage[utf8]{inputenc}
\usepackage{pseudocode}

\usepackage{amsthm} 
\usepackage{amsmath} 
\usepackage{amssymb}


\usepackage{verbdef}% http://ctan.org/pkg/verbdef


\setlength{\headheight}{8mm}
\setlength{\oddsidemargin}{7mm}
\setlength{\topmargin}{-15mm}        
\setlength{\textwidth}{145mm}
\setlength{\textheight}{230mm}       

\title{\center{Stratosphere hackathon}}

\date{}
\author{}

\begin{document}

%\maketitle

\section{Install Stratosphere}

\begin{verbatim}
git clone https://github.com/stratosphere/stratosphere.git

cd stratosphere

mvn -DskipTests clean package #it can take some time
\end{verbatim}

Now you have a Stratosphere distribution in your stratosphere-dist/target/stratosphere-dist-0.4-SNAPSHOT-bin/stratosphere-0.4-SNAPSHOT directory.

\section{Compile and run}

How to compile the WordCount example? How to run it in local mode?

\begin{verbatim}
stratosphere-dist/target/stratosphere-dist-0.4-SNAPSHOT-bin/stratosphere-0.4-SNAPSHOT/bin/start-local.sh

javac -cp 'stratosphere-dist/target/stratosphere-dist-0.4-SNAPSHOT-bin/stratosphere-0.4-SNAPSHOT/lib/*' pact/pact-examples/src/main/java/eu/stratosphere/pact/example/wordcount/WordCount.java pact/pact-examples/src/main/java/eu/stratosphere/pact/example/util/AsciiUtils.java

jar cvf wordcount.jar -C pact/pact-examples/src/main/java .
\end{verbatim}

Get some test data:

\begin{verbatim}
wget -O ~/hamlet.txt http://www.gutenberg.org/cache/epub/1787/pg1787.txt
\end{verbatim}

Run the job:

\begin{verbatim}
stratosphere-dist/target/stratosphere-dist-0.4-SNAPSHOT-bin/stratosphere-0.4-SNAPSHOT/bin/pact-client.sh run -j wordcount.jar -c eu.stratosphere.pact.example.wordcount.WordCount -a 1 file:///home/<user>/hamlet.txt file:///home/<user>/wordcount_output
\end{verbatim}


\section{Running jobs on the cluster}

\verbdef{\vtext}{hadoop01 }
From machine \vtext you can submit jobs to the Stratosphere cluster. This is the path of the Stratosphere:

\begin{verbatim}
/mnt/scratch1/tempdata/stratosphere-ozone/stratosphere-master/stratosphere-dist/target/stratosphere-dist-0.4-ozone-SNAPSHOT-bin/stratosphere-0.4-ozone-SNAPSHOT/
\end{verbatim}

This is the path of the Hadoop (you can use this for handling the hdfs):

\begin{verbatim}
/mnt/scratch1/tempdata/stratosphere-ozone/hadoop/hadoop-0.20.203.0/
\end{verbatim}

Submitting a job into the Stratosphere cluster you can give the path on the hdfs the following way:

\begin{verbatim}
hdfs://hadoop01:8052/...
\end{verbatim}

\section{Training exercises}

Suppose that we represent a vector $v$ as PACT records $(i, v_i)$. And we represent
a matrix $M$ as PACT records $(i, j, m_{ij})$.

\begin{itemize}
  \item compute the scalar product of two vectors: $u^T v$
  \item compute the dyadic product of two vectors: $v u^T$
  \item compute the product of a vector and a matrix: $Av$
  \item compute the product of two matrices: $AB$
  \item Given matrices $P, Q$ and $R$, compute $||PQ - R||_F$ (where $||\cdot||$ is the Frobenius norm, i.\ e.\  the sum of
    the squares of the elements of the matrix)
\end{itemize}

\section{ALS}

\noindent
Input: $R \in \mathbb{R}^{|I| \times |U|}$, $r_{ui}$, where $(u,i) \in A$, k dimension, (and the number of iterations: N)

\noindent
Output: $P,Q$ (so that $PQ$ should approximate $R$)

\noindent
Algorithm steps: \\

\begin{pseudocode}{generateRandomQ}{|I|, k}
  Q = \mbox{ a random matrix of size } |I| \times k
\end{pseudocode},

\begin{pseudocode}{updateP}{R, Q}
\FOREACH  u \in U \DO
  p_u := (\sum_{i \in I_u} q_i q_i^T)^{-1} (\sum_{i \in I_u} r_{ui}q_i)
\end{pseudocode}

where $I_u = \{i: (u,i) \in A\}$

\begin{pseudocode}{updateQ}{R, P}
\FOREACH i \in I \DO
  q_i := (\sum_{u \in U_i} p_u p_u^T)^{-1} (\sum_{u \in U_i} r_{ui}p_u)
\end{pseudocode}

where $U_i = \{u: (u,i) \in A\}$

\begin{pseudocode}{ALS}{R, k, N}
  Q = generateRandom(|I|, k) \\
  \FOR i \GETS 0 \TO N \DO
    \BEGIN
      P:=updateP(R, Q) \\
      Q:=updateQ(R, P)
    \END
\end{pseudocode}

\end{document}

